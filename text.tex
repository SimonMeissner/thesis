\documentclass[12pt, twoside]{article}
% Packages
\usepackage[utf8]{inputenc} % Input encoding (choose the appropriate encoding for your system)
\usepackage[T1]{fontenc} % Font encoding for better output
\usepackage{setspace} % For custom line spacing
\usepackage{graphicx, float} % For including images
\usepackage{amsmath, amssymb} % Math packages
\usepackage{natbib} % For bibliography management
\usepackage{hyperref} % For hyperlinks and clickable references
\graphicspath{{images/}}

% Customization ---------

% GEOMETRY
% sets paper size, margins and a parameter that's a good idea
\usepackage[letterpaper, top=1in, bottom=1.0in, left=1.2in, right=1.2in, heightrounded]{geometry}

% line height
\renewcommand{\baselinestretch}{1.15}

% parindent parskip
\setlength{\parindent}{0pt}
\setlength{\parskip}{0.8em}

% title customization


% -----------------------
% Tile page
\title{Designing interaction techniques for the comparison of spatial entities in the context of geodashboards}
\author{Simon Meißner}
\date{August 2023}

% Begin the document
\begin{document}

% Title page
\maketitle
\newpage
% Abstract
\begin{abstract}
% Your abstract goes here
\end{abstract}
\newpage
% Table of Contents
\tableofcontents

% List of Figures (if applicable)
% \listoffigures

% List of Tables (if applicable)
% \listoftables

\newpage

\section{Introduction}
The growing usage of dashboards to represent data across a range of different 
fields suggests a need for research on layout and design features of dashboards 
and their influence on the user experience \citep*{Yigitbasioglu.2012,Sarikaya.2018}.

Previous research has shown that there is no one-fits-all solution and it has to be
experimented with different design decisions in different scopes and purposes.
An interesting approach to narrow down the area of investigation is to classify 
user interaction in a dashboard application for exploration and/or analysis of the
dataset \citep{Wehrend.1990}. This doesn't exclude the field of geovisualizations and
geodashboards. There are many different perspectives that all reasonably try to define
taxonomies and or classification models that all have their application for
spatio-temporal data exploration and/or analysis
\citep*{Andrienko.2003,Crampton.2002}.

But Roth has shown that a functional taxonomy of interaction primitives can be
empirically derived. He identified general tasks users want to accomplish (objective
primitives) \citep{Roth.2013}.

An interaction techniques as broadly defined in the Computer Science Handbook from 2004
is "the fusion of input and output, consisting of all hardware and software elements, 
that provides  a way for the user to accomplish a task.” \citep*{Hinckley.2004}. In the
context of geovisualizations and geodashboards interaction techniques have been
researched \citep*{Keim.2005,Lobo.2015,Roth.2013,vanTonder.2011}.
Roth also describes an interaction technique in the context of geovisualizations as the
functionality of an given interface and the procedures of manipulating its
visualizations \citep{Roth.2013}.

Instead of dealing with layout and design features directly, this work will focus on
interaction techniques as they also imply design decisions. As we have seen it is
reasonable to focus on a specific usecase in geodashboards. This work will deal with
on the derived objective primitive of \textit{comparison} from Roth's work. But not
only Roth speaks about comparison. Wehrend describes \textit{compare} as a seperate
operation class in visualization problem \citep{Wehrend.1990} and Brehmer et al.
speak of comparison as a low level visualization task \citep*{Brehmer.2013}. In the
scope of geovisualizations Crampton identified \textit{compare} as an interactivity task
\citep{Crampton.2002} and Gorte and Degbelo argue that \textit{comparison} is a basic
task that is relevant in exploratoy and confirmatory analysis \citep{Gorte.2022}.

Distingiushes between two dimensions of comparison. Comparisons of
variables or projections, and comparison of subset of the data \citep*{Buja.1996}.


\section{Literature Review}
% Your literature review goes here

\section{Methodology}
% Your methodology goes here

\section{Results}
% Your results go here

\section{Discussion}
% Your discussion goes here

\section{Conclusion}
% Your conclusion goes here

% Bibliography
\bibliographystyle{plain}
\bibliography{references} % Replace "references.bib" with your bibliography file

% Appendix (if applicable)
% \appendix
% \chapter{Appendix Title}
% Your appendix content goes here

\end{document}