\documentclass[12pt, twoside]{article}
% Packages
\usepackage[utf8]{inputenc} % Input encoding (choose the appropriate encoding for your system)
\usepackage[T1]{fontenc} % Font encoding for better output
\usepackage{setspace} % For custom line spacing
\usepackage{graphicx, float} % For including images
\usepackage{amsmath, amssymb} % Math packages
\usepackage{natbib} % For bibliography management
\usepackage{hyperref} % For hyperlinks and clickable references
\graphicspath{{images/}}

% Customization ---------

% GEOMETRY
% sets paper size, margins and a parameter that's a good idea
\usepackage[letterpaper, top=1in, bottom=1.0in, left=1.2in, right=1.2in, heightrounded]{geometry}

% line height
\renewcommand{\baselinestretch}{1.15}

% parindent parskip
\setlength{\parindent}{0pt}
\setlength{\parskip}{0.8em}

% title customization


% -----------------------
% Tile page
\title{Modeling the performance of  interaction techniques for the comparison of spatial entities in the context of geo-dashboards}
\author{Simon Meißner}
\date{November 2023}

% Begin the document
\begin{document}

% Title page
\maketitle
\newpage
% Abstract
\begin{abstract}
% Your abstract goes here
\end{abstract}
\newpage
% Table of Contents
\tableofcontents

% List of Figures (if applicable)
% \listoffigures

% List of Tables (if applicable)
% \listoftables

\newpage

\section{Introduction}
The growing usage of dashboards to represent data across a range of different 
fields suggests a need for research on layout and design features of dashboards 
and their influence on the user experience. Previous research has shown that there is
no one-fits-all solution for the design of dashboards \citep*{Yigitbasioglu.2012,Sarikaya.2018}.

This existing agreement that general design recommendations for dashboards are not sufficient
and/or possible ask for a breakdown of components that constitute dashboards. One approach is
to identify user interactions that are possible when visualized data is explored or analysed.
The field of geovisualizations and geodashboards is enriched with many different perspectives that
all try to define taxonomies and or classifications models for possible user interactions
\citep*{Andrienko.2003,Crampton.2002}. Many have their reasonable own application for
spatial-temporal information visualization. But the minority of these classifications and
taxonomies are empirically-derived. The proposed framework of Roth represents an exception.
He has shown that a functional taxonomy of interaction primitives can be empirically derived.
He identified general tasks users want to accomplish (objective primitives) \citep{Roth.2013}.
Besides narrowing down the scope to one of Roths derived objective primitives this work will also
look at this topic from the perspective of different interaction techniques.

An interaction technique as broadly defined in the Computer Science Handbook from 2004
is "the fusion of input and output, consisting of all hardware and software elements, 
that provides a way for the user to accomplish a task." \citep*{Hinckley.2004}. Input describes
all sensed information about the physical environment to the computer. Output from computers on the
other hand include any emission or modification to the physical environment. In the
context of geovisualizations and geodashboards, interaction techniques have been
researched \citep*{Keim.2005,Lobo.2015,vanTonder.2011}.
Roth also describes an interaction technique in the context of geovisualizations as the
functionality of an given interface and the procedures of manipulating its
visualizations \citep{Roth.2013}.

This work will deal with the derived objective primitive of \textit{comparison} from Roth's work.
But not only Roth writes about comparison. Wehrend describes \textit{compare} as a seperate
operation class in visualization problem \citep{Wehrend.1990} and Brehmer et al.
speak of comparison as a low level visualization task \citep*{Brehmer.2013}. In the
scope of geovisualizations Crampton identified \textit{compare} as an interactivity task
\citep{Crampton.2002} and Gorte and Degbelo argue that \textit{comparison} is a basic
task that is relevant in exploratoy and confirmatory analysis \citep{Gorte.2022}.
Buja distingiushes between two dimensions of comparison. The first
describes the goal of comparing different variables or projections of the whoel dataset.
The second describes the goal of comparing subsets of the whole dataset against each
other \citep*{Buja.1996}. This work will only focus on the latter. 

We will examine two broadly used interaction techniques: \textit{filtering} and
\textit{highlighting} \citep*{Keim.2005,Roth.2013}. Keim et al. describe \textit{filtering} as a
combination of selection and view enhancement and Roth attributes filtering to identifying matches
from user-defined conditions. The literature often use the term \textit{brushing} 
to describe highlighting but mean the process of visualy emphasizing one or more features from the
whole dataset. For the rest of this work we will use the term \textit{highlighting}. Keim et al. state
that \textit{highlighting} is often combined with linking which describes the process of selected data
beeing communicated to other views of the data. They follow one of the proposed user strategies
\textit{Select Subset} from Gleichner \citep*{Gleicher.2018}.

In this work we will investigate how these interaction techniques influence user performance
in the context of comparing subsets in geodashboards. Because the interaction technique is
by far not the only variable that can be changed, we also want to observe the influence of
different variables on the user performance. To provide a starting point backed with emperical
data we want to derive a mathematical model that should display user performance in dependence
from different variables which are described later. Therefore we can infer two research
questions for this work:
\begin{enumerate}
    \item Which mathematical models best describe user performance during the comparison of spatial entities
    in the context of geodashboards?
    \item Which interaction technique best supports the task of comparison in the context of
    geodashboards? 
\end{enumerate}

To answer these questions we conducted a user study in which participants try to answer
questions with the goal of finding differences and/or similarities of subsets
of spatial-/temporal datasets. To answer the
questions they are using a specially builded digital web-prototype with six different dashboard
variants. The dashboards vary in their interaction technique and some render additional views
utilizing \textit{explicit encoding} as it is defined as one of the basic designs for visual
comparison \citep*{Gleicher.2018}. The goal of the experiment is to collect data about the user
performance. We have defined user performance to include the time it takes to answer questions
and the accuracy of that answer. After that we want to use that data to derive mathematical
models that best approximate answer time and accuracy during the comparison of features in
geodashboards. With special interest for the differences between the selected interaction
techniques. We want to learn about the different factors we have included and how they
influence answer time and accuracy in this setting.

To get an idea about core subjects in this work section 2 will touch upon related work. We
will give an outline about geo-dashboards and summarize findings from empirical studies in that
regard. We will also give an short introduction to multiple coordinated because we are utilizing
such a system in our prototype. We will also speak more about \textit{filtering} and
\textit{highlighting} as interaction techniques. Lastly we will have a deeper look into what to
consider when designing visualization for comparison.
Section 3 will describe key concepts of the mathematical models and what constitues them.
In Section 4 details about the digital web-prototype and how it was built are presented. How the
experiment was designed and what factors that possibly influence comparison performance were considered
are covered in section 5. Section 6 will present the results of the experiment and propose our found
mathematical models that showed the best testing results.
As this experiment only covers a selection of possible factors that possible influence difficulty and
accuracy during the comparison of features in geodashboards, this work should be a starting point for
further research. Because comparison can be of many different kinds and cover different scopes this work
also opens the door for more reseach in different comparison settings. Section 7 discusses such limitations
in depth, how our research questions can be answered and how our findings can be transferred to other domains.
Lastly we will sumarize our key-learnings and propose future work that has to be done in section 8  .


\section{Related Work}
% Your literature review goes here
\subsection*{Comparison}
Gleichner writes about four considerations when visualizing comparison \citep*{Gleicher.2018}.
At first we have to identify our comparison elements. Because every comparison task in our
study focuses on comparing two or three features of the dataset we can descibe our targets as
'explicit targets' as every item is known and already available. From Gleichners proposed
actions on relationships between targets our comparison task fall into the \textit{Identify}
and \textit{Measure/Quantify/Summarize} categories. Second we have to identify comparitive
challenges. The number of targets should not add much complexity as we are only using two
or three targets. Because we are using timeseries with only one observed variable the
complexity of each individual item is also fairly low. As we are only identifying and
measuring direct differences or differences between differences of targets the complexity
of the relationships is low to moderate. Gleichner next proposes to deal with a scalability
strategie. To reduce scale challenges the strategies of \textit{Select Subset} and 
\textit{Summarize Somehow} are utilized. In all dashboard variants either \textit{filtering} or
\textit{highlighting} as an interaction technique is used which help with scale because subsets
are created. In some variants additional views are rendered that already encode the
difference of two targets which summarize a relation. Lastly we have to consider design
visualizations. Across all dashboard variants all three basic visual designs for comparison
are utilized as each have their benefits and drawbacks. Because we deal with temporal data
all graph views are utilizing \textit{superposition}. To reduce scalability problems either
\textit{filtering} or \textit{highlighting} are utilized as already mentioned. Our table views
use a combination of \textit{super} - and \textit{juxtapostion} as showing each datapoint in the
same place would hinder readability. Finally because our actions on the targets include comparing
differences between targets we included \textit{explicit encoding} in some dashboard variants.

\section{Mathematical models}

\section{Digital Web-prototype}

\section{Methodology}

\section{Evaluation}

\section{Discussion}

\section{Conclusion}

% Bibliography
\bibliographystyle{plain}
\bibliography{references} % Replace "references.bib" with your bibliography file

% Appendix (if applicable)
% \appendix
% \chapter{Appendix Title}
% Your appendix content goes here

\end{document}