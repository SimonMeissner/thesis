\chapter{Related Work}
In this chapter we want to summarize existing research with regard to our two interaction techniques:
\textit{highlighting} and \textit{filtering}. First we will look at how the use of \textit{highlighting}
and \textit{filtering} in the context of geo-dashboards is analysed. Second, we will also examine the
interaction techniques in the context of multiple coordinated views because this principle plays a central
role in our web prototype.

\section{Interaction techniques \&\ Geo-dashboards}
Geo-dashboards (or geospatial dashboards) provide insights on geographically referenced data \citep*{Bernasocchi.2012}.
Jing et al. extend the definition of Badard and Dubé(\citep*{ThierryBadard.2009}) of geo-dashboards to be "a web-based interactive interface that is supported by a platform combining
mapping, spatial analysis, and visualization with proven business intelligence tools" \citep*{Jing.2019}.

Interaction techniques have been researched extensivly in the context of geo-dashboards \citep*{Keim.2005,Lobo.2015,vanTonder.2011}.
For example Lobo et al. researched different interaction technique for layer comparison in geo-visualizations (\citep*{Lobo.2015}) and
VanTonder et al. investigated new tilting interaction techniques when performing typical tasks on mobile map-based applications \citep*{vanTonder.2011}.
Ahonen-Rainio and Kraak argue that interactive techniques like \textit{highlighting} (brushing) and \textit{filtering} supports the user
to interpret geo-visualizations \citep*{AhonenRainio.2005}.
Keim et al. state that an information visualization is constituted of three dimensions. The type of data that is visualized, the visualization technique used
and the interaction technique used. Any combination of these three dimensions is conceivable. They sub categorize interaction techniques further into
view enhancement methods, navigation techniques and selection techniques in which they list techniques for \textit{filtering} and or \textit{highlighting}. Further they state
\textit{highlighting} (brushing) to be one specifc interaction selction technique that is often combined with linking. Meaning the use of multiple views to display the
data in several ways with the same features in each view visualy emphasized through \textit{highlighting}. They state that more subsequent geo-visualizations incorporate the
use of \textit{highlighting} and linking \citep*{Keim.2005}. They also mention interactive \textit{filtering} to be a combination of selection and view enhancement. When exploring large
data sets it is important to interactivly select subsets of interest. Techiques that offer a more direct and interactive \textit{filtering} a as the traditional querying are
preferred more and more. \citep*{Keim.2005}. Ahonen-Rainio and Kraak argue that interactive techniques like \textit{highlighting} (brushing) and \textit{filtering} supports the user
to interpret geo-visualizations \citep*{AhonenRainio.2005}. Roberts states that users may want to directly manipulate query results in the visualization besides using button
and sliders. Roberts also summarizes trends and present current example applications that implement the idea of \textit{highlighting}
in comination with linking multiple views but also states that the idea is still underused in geo-visualizations \citep*{Roberts.2005}.

\section{Interaction techniques \&\ Multiple coordinated views}
Multiple Coordinated Views (in the following abbreviated with ’MCV’) is a specific exploratory visualization technique that allows users to explore data.
It consists of multiple views that all encode the same data in different representations. Interactions and
operations of the user are managed and synchronised between views \citep*{Roberts.722007722007}. Buja et al. argue that
beeing able to pose queries graphically and viewing the response in the same visual field is a fundamental component
\citep*{Buja.1996}. The principle is extensivly accepted and researched and proven to increase user performance, discovery
of unforeseen relationships and unification of the desktop \citep*{North.1998}. Regarding multidimensional geographic data they are
commonly used to facilitate the visual exploration to assist decision-making processes \citep*{Andrienko.2020}.
Buja et al. even include MCV in their taxonomy for data visualization. They even mention \textit{highlighting}, one of our two researched
interaction techniques, to often be a substantial part of MCV as it is one way pose queries about subsets graphically \citep*{Buja.1996}.

Because of its popularity 20-30 years ago much of previous research in the field of MCV focused on
scatterplot matrices \citep*{Carr.1987, Becker.1987}. Lawrence et al. used a specfic tool for the exploratory analysis of systems biology data
and showed how \textit{highlighting} is an effective technique for discovering outliers \citep*{Lawrence.2006}.
Carr et al. argue that \textit{highlighting} is the most common interaction technique in scatterplot matrices
when working with subsets \citep*{Carr.1987, Becker.1987}. On the other hand they argue that if the subsets becomes larger
another approach may be more suitable. They called it \textit{specify, then compute}.
The idea is that a selection region is defined, similar to the \textit{highlighting} procedure, then the subset
is computed and finally a new display is rendered that only contains the subset. This describes our more
modern idea of \textit{filtering}.
Some contributions on interaction techniques in MCV systems define \textit{filtering} as a type of
\textit{highlighting} (\textit{brushing}) \citep*{Ward.}. Generally it seemed that our idea of \textit{filtering} is seen more as a specific
realization of \textit{highlighting}. They are not compared against each other. As an exeption Andrienko et al. enumerate both \textit{highlighting}
and \textit{filtering} to be mechanisms to realize coordination accross views. We could not find any qualitative or quantitative comparisons between
both techniques in any regard.