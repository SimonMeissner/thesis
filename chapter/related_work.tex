\chapter{Related Work}
In this chapter we want to summarize existing research with regard to our two interaction techniques:
\textit{highlighting} and \textit{filtering}. First we will look at how the use of \textit{highlighting}
and \textit{filtering} in the context of geodashboards is analysed. Second, we will also examine the
interaction techniques in the context of multiple coordinated views because of its central role in our web-prototype.
\section{Interaction techniques \&\ Geo-dashboards}


Interaction techniques have been researched extensivly in the context of geo-dashboards.
Lobo et al. researched different interaction technique for layer comparison in geo-visualizations \citep*{Lobo.2015}.
VanTonder et al. investigated new tilting interaction techniques when performing typical tasks on mobile map-based applications \citep*{vanTonder.2011}.

Ahonen-Rainio and Kraak argue that interactive techniques like highlighting (brushing) and filtering supports the user
to interpret geovisualizations \citep*{AhonenRainio.2005}.
Keim et al. state that an information visualization is constituted of three dimensions. The type of data that is visualized, the visualization technique used
and the interaction technique used. Any combination of these three dimensions is conceivable. They sub categorize interaction techniques further into
view enhancement methods, navigation techniques and selection techniques in which they list techniques for filtering and or highlighting. Further they state
brushing (highlighting) to be one specifc interaction selction technique that is often combined with linking. Meaning the use of multiple views to display the
data in multiple ways with the same features in each view visualy emphasized throw highlighting. They state that more subsequent geovisualizations incorporate the
use of highlighting and linking \citep*{Keim.2005}. They also mention interactive filtering to be a combination of selection and view enhancement. When exploring large
data sets it is important to interactivly select subsets of interest. Techiques that offer a more direct and interactive filtering a as the traditional querying are
preferred more and more. \citep*{Keim.2005}. Ahonen-Rainio and Kraak argue that interactive techniques like highlighting (brushing) and filtering supports the user
to interpret geovisualizations \citep*{AhonenRainio.2005}. Roberts states that users may want to directly manipulate query results in the visualization besides using button
and sliders. Roberts also summarizes trends and present example applications that implement the idea of \textit{highlighting}
in comination with linking multiple views but also states that the idea is still underused in geovisualizations \citep*{Roberts.2005}.


see legends of the dashboard
\section{Interaction techniques \&\ Multiple coordinated views}
Multiple Coordinated Views (in the following abbreviated with ’MCV’) is a specific exploratory visualization technique that allow users to explore data.
It consists of multiple views that all encode the same data in different representations. Interactions and
operations of the user are managed and synchronised between views \citep*{Roberts.722007722007}. Buja et al. argue that
beeing able to pose queries graphically and viewing the response in the same visual field is a fundamental component
\citep*{Buja.1996}. The principle is extensivly accepted and researched and proven to increase user performance, discovery
of unforeseen relationships and unification of the desktop \citep*{North.1998}. Buja et al. even include it in their taxonomy
for data visualization. They even mention \textit{highlighting}, one of our two researched interaction techniques,
to often be a substantial part of MCV as it is one way pose queries about subsets graphically \citep*{Buja.1996}.

Because of its popularity 20-30 years ago much of previous research in the field of MCV focused on
scatterplot matrices \citep*{Carr.1987, Becker.1987}. Lawrence et al. used a specfic tool for the exploratory analysis of systems biology data
and showed how \textit{highlighting} is an effective technique for discovering outliers \citep*{Lawrence.2006}.
Carr et al. argue that \textit{highlighting} is the most common interaction technique in scatterplot matrices
when working with subsets \citep*{Carr.1987, Becker.1987}. On the other hand they argue that if the subsets becomes larger
another approach may be more suitable. They called it \textit{specify, then compute}.
The idea is that a selection region is defined, similar to the highlighting procedure, then the subset
is computed and finally a new display is rendered that only contains the subset. This describes our more
modern idea of \textit{filtering}. Some contributions on interaction techniques in MCV systems define \textit{filtering} as a type of
\textit{highlighting} (\textit{brushing}) \citep*{Ward.}. Generally it seems our idea of \textit{filtering} is not much researched in the context
of MCV or seen as similar to \textit{highlighting} techniques. 

\section{Comparison}
Gleicher writes about four considerations when visualizing comparison \citep*{Gleicher.2018}.
At first we have to identify our comparison elements. Because every comparison task in our
study focuses on comparing two or three features of the dataset we can descibe our targets as
'explicit targets' as every item is known and already available. From Gleichers proposed
actions on relationships between targets our comparison task fall into the \textit{Identify}
and \textit{Measure/Quantify/Summarize} categories. Second we have to identify comparitive
challenges. The number of targets should not add much complexity as we are only using two
or three targets. Because we are using timeseries with only one observed variable the
complexity of each individual item is also fairly low. As we are only identifying and
measuring direct differences or differences between differences of targets the complexity
of the relationships is low to moderate. Gleicher next proposes to deal with a scalability
strategie. To reduce scale challenges the strategies of \textit{Select Subset} and 
\textit{Summarize Somehow} are utilized. In all dashboard variants either \textit{filtering} or
\textit{highlighting} as an interaction technique is used which help with scale because subsets
are created. In some variants additional views are rendered that already encode the
difference of two targets which summarize a relation. Lastly we have to consider design
visualizations. Across all dashboard variants all three basic visual designs for comparison
are utilized as each have their benefits and drawbacks. Because we deal with temporal data
all graph views are utilizing \textit{superposition}. To reduce scalability problems either
\textit{filtering} or \textit{highlighting} are utilized as already mentioned. Our table views
use a combination of \textit{super} - and \textit{juxtapostion} as showing each datapoint in the
same place would hinder readability. Finally because our actions on the targets include comparing
differences between targets we included \textit{explicit encoding} in some dashboard variants.