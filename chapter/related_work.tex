\chapter{Related Work}
% Your literature review goes here
\section*{Comparison}
Gleichner writes about four considerations when visualizing comparison \citep*{Gleicher.2018}.
At first we have to identify our comparison elements. Because every comparison task in our
study focuses on comparing two or three features of the dataset we can descibe our targets as
'explicit targets' as every item is known and already available. From Gleichners proposed
actions on relationships between targets our comparison task fall into the \textit{Identify}
and \textit{Measure/Quantify/Summarize} categories. Second we have to identify comparitive
challenges. The number of targets should not add much complexity as we are only using two
or three targets. Because we are using timeseries with only one observed variable the
complexity of each individual item is also fairly low. As we are only identifying and
measuring direct differences or differences between differences of targets the complexity
of the relationships is low to moderate. Gleichner next proposes to deal with a scalability
strategie. To reduce scale challenges the strategies of \textit{Select Subset} and 
\textit{Summarize Somehow} are utilized. In all dashboard variants either \textit{filtering} or
\textit{highlighting} as an interaction technique is used which help with scale because subsets
are created. In some variants additional views are rendered that already encode the
difference of two targets which summarize a relation. Lastly we have to consider design
visualizations. Across all dashboard variants all three basic visual designs for comparison
are utilized as each have their benefits and drawbacks. Because we deal with temporal data
all graph views are utilizing \textit{superposition}. To reduce scalability problems either
\textit{filtering} or \textit{highlighting} are utilized as already mentioned. Our table views
use a combination of \textit{super} - and \textit{juxtapostion} as showing each datapoint in the
same place would hinder readability. Finally because our actions on the targets include comparing
differences between targets we included \textit{explicit encoding} in some dashboard variants.