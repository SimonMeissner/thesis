
\newlist{statements}{enumerate}{1}
\setlist[statements,1]{label=\arabic*., left=0.5cm}
\chapter{Mathematical models}
Because our dataset is limited, we have decided to use linear regression to try to derive a function that explains how
a possible connection between user performance and our independent variables looks. Machine learning requires much more
data to be effective.

First we want to look at the user performance in terms of their time used to answer the question (in the following
abbreviated with 'answer time'). Because we have such limited data available and because it simplifies the analization
process we decided to include all datapoints, even if the answer was not correct.
Inspired by Fitts‘ Law, which is often used in HCI (human computer interaction) we focus on one possible aspect of answer
time. An index that describes the level of difficulty of the task to which the user has given an answer. We now want to
look how all our independent variables influenced the DifficultyIndex.
\section{Assumptions}
One of the rationales of MCV is that more views on the same data, help with data exploration (if applied correctly).
Because comparison is a type of data exploration we can assume that more views also help increase user performance in tasks
that involve comparison. We can derive:
\begin{statements}
    \item A higher number of views on the same data will reduce the DifficultyIndex.
\end{statements}
The suitability of views change across different contexts. One context for example is what task is tried to be solved
using the view. In our experiment, we used a total of four different task types, which are represented by the four question
types. Half of the questions ask for comparison of numerical(\textit{attribute in space}) information and the other half
asks for comparison of temporal(\textit{space in time}) information. Lohse et al. classify visual representations(views)
depending on the type of information conveyed. They classified 11 different view types and empircally derived likert
scores(1-9) for every type of information \citep*{Lohse.1994}. Among other things they classified \textit{time charts} and 
\textit{tables} when dealing with \textit{temporal} or \textit{numerical} information. We use these empircal likert scores
to quantify the quality of the views used when answering a specific question. Since each interaction technique uses at least
one table and one time chart and we cannot predict which view the user will use, we need to calculate the mean likert score
of both views for a given question type. In the experiment the question types have a second dimension. Questions could
either be of type \textit{identify} or \textit{measure}. In our mathematical model we do not distinguish between these
two types because we lack research that would help quantify view quality in those regards. We can conclude with the
following assumption:
\begin{statements}[resume]
    \item A higher combined encoding quality of the views in relation to the type of question will reduce the
    DifficultyIndex
\end{statements}
\begin{statements}[resume]
    \item A higher number of distractors will increase the DifficultyIndex. Distractors are elements in the user interface
    that impede the user to find the answer
\end{statements}

$
F_{difficulty} = \frac{1}{n_{views}} + \frac{n_{targets}}{1} + Difficulty_{question-type} + Difficulty_{interaction-technique}
$