\chapter{Conclusion}
This work investigated which mathematical models best describe user performance during the comparison of spatial entities in the context
of geo-dashboards. We investigated the use of different \textit{interaction techniques} in this context and wanted to know what technique
supports the task of comparison best. A web prototype was built to conduct an explorative study that recorded user performance when answering 
comparison questions. Based on previous research we aggregated our four selected study variables into mathematical constructs
which should help us find continuous descriptive models regarding the user performance during comparison tasks. Overall we could not find
any sufficiently convincing model that describes our recorded user performances. However further analysis could confirm previous assumptions
about the number of targets that are asked to be compared influencing the task difficulty. We also confirmed that the type of comparative
question that is asked has a significant influence on the user performance. Our analysis showed no significant differences in user performance
across different interaction techniques. Subjective feedback suggests that performance differences may exist but require better study designs in the
future.
