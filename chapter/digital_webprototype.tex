\chapter{Digital web prototype}
The prototype is created with the open-source web development framework next.js, which offers react-based javascript- or
typescript-based applications with server-side rendering and static website generation. It is mostly selected because of its
developer-friendly ecosystem, especially regarding deployment. Another reason for this stack is the existing experience
with react-based typescript and the lightweight, modern setup. The whole application makes use of a few react libraries.
The most important being the popular "react-leaflet" library which drastically simplifies the use of leaflet maps in react
applications. Other important libraries are: "zustand" to help with coordination, "recharts" to help build timelines, and
"tailwindcss" to simplify the design process of the application. Because the experiment is designed to be held online the
application needs to be deployed. For that, we use the free deployment plan of vercel.com. Because the application is only
using non-personal static data there is no need for implementing a backend or an authentication system. Although the prototype
takes responsive behavior into account the screen dimension of the user's device should not fall below 1280x720px. 
\section{Data} \label{dataSection}
The data used in the web prototype consists of two components. First, we implement georeferenced data of the states of Germany. In terms of
performance, low-resolution geojson data with a reference scale of 1:5000000 is used. We utilize open-source data
from the German Federal Agency for cartography and geodesy \citep*{gdz.bkg}. The questions in the study are about static
spatio-temporal datasets focusing on changes in thematic properties expressed through values of attributes. Meaning
qualitative changes with numeric characteristics. According to Andrienko et al., this is one of three types of spatio-temporal
data \citep*{Andrienko.2003}. On the other hand, spatio-temporal datasets can have existential changes, features appear and
disappear over time, or they can have changes in their spatial properties. Because we want to minimize learning effects
when using the data and because every participant is going to answer eight questions, we implement four different
datasets. The dataset is switched after every question using the "dataset control" which will be explained later. All four datasets
have the same structure. They are about qualitative numeric changes of one marker of all 16 states of Germany and cover
the years from 2008 to 2022.
\section{Interaction techniques}
In addition to the methods already presented: \textit{filtering} and \textit{highlighting}, we decided to investigate a third
method. It is a modification of the highlighting method and is described by Andrienko et al., among others. When multiple subsets are selected for
comparison, they are visually highlighted using \textit{different} colors to indicate several subsets \citep*{Andrienko.2020}. In classic highlighting, all selected
subsets are highlighted in the \textit{same} color even if they represent different subsets. For the rest of this work, we will distinguish
highlighting using the same color and highlighting using different colors by naming the first \textit{highlighting\_1} and the second \textit{highlighting\_2}.

\section{Layout \&\ Function}
In addition to the static data, the entire application consists of a simple frontend that serves six different non-scrollable
geo-dashboard variants. Always two of the six variants implement the same interaction technique. Variant 1 \&\ 4 use
\textit{filtering}, variant 2 \&\ 5 use \textit{highlighting\_1} and variant 3 \&\ 6 use \textit{highlighting\_2}
(see Table \ref{dashboard_variant_table}). When accessing the prototype the first variant is rendered. The other variants are
reachable over a tab-based navigation bar at the top of the screen. A seventh tab is implemented to learn about the application
and its purpose. Each variant is covered with an openstreet map that is centered on Germany stretching over the whole page. All
states of Germany are rendered using the borders as polygon representations. Every polygon used the same
blue fill color. All other elements in the dashboards are placed on top of this map. On every variant, two additional views
are also rendered on top of each other, taking up the left third of the screen. The time chart in the top-left corner of the
screen visualizes the spatial temporal data by rendering 16 lines, each representing one of the 16 states of Germany. Our time chart
follows the principle of \textit{superposition} as it renders all features in the same chart \citep*{Gleicher.2018}. The
bottom-left view visualizes a two-dimensional table connecting each state with each year containing a numerical value.
A table uses a combination of \textit{super-} and \textit{juxtapostion} which are two of the three basic visual designs for comparison
from Gleicher \citep*{Gleicher.2018} The "dataset control" and another button, which will be termed "comparison control" are positioned
in the middle third of the screen close to the navigation bar at the top. The comparison control offers the ability to select two or three of the 16
states for comparison. After selecting the states and confirming the comparison process the application enters "comparison mode" 
where three of the six variants render an additional time chart and an additional table view. In both views calculated differences
between two states of the values of the selected states are displayed. The time chart again consists of lines representing the temporal
evolution of the differences, and the table view contains the numerical differences as simple numerical values for every year.
Those are the already mentioned views that use \textit{explicit encoding} as defined by Gleicher \citep*{Gleicher.2018}.
\begin{longtable}{| p{0.15\linewidth} | p{0.50\linewidth} | p{0.25\linewidth}|}
    \hline
    \textbf{Variant} & \textbf{Interaction technique} & \textbf{Number of views} \\
    \hline
    \endfirsthead
    \multicolumn{3}{l}{{\textit{Continued from previous page}}} \\
    \hline
    \textbf{Variant} & \textbf{Interaction technique} & \textbf{Number of views} \\
    \hline
    \endhead
    \hline \multicolumn{3}{r}{{\textit{Continued on next page}}} \\
    \endfoot
    \hline
    \caption{This table describes all dashboard variants visualized in the web prototype. The number of views it rendered in "comparison mode" and the interaction technique is displayed \label{dashboard_variant_table}}\\
    \endlastfoot
    1 & \parbox{\linewidth}{\vspace{4pt} filtering} & 2\\
    \hline
    2 & \parbox{\linewidth}{\vspace{4pt} highlighting\_1} & 2\\
    \hline
    3 & \parbox{\linewidth}{\vspace{4pt} highlighting\_2} & 2\\
    \hline
    4 & \parbox{\linewidth}{\vspace{4pt} filtering} & 4\\
    \hline
    5 & \parbox{\linewidth}{\vspace{4pt} highlighting\_1} & 4\\
    \hline
    6 & \parbox{\linewidth}{\vspace{4pt} highlighting\_2} & 4\\
\end{longtable}
After starting the comparison process both views on the left are replaced with other views depending on the interaction technique. \textit{Replace}
is one of the three common operational models visual interfaces use on parameter change \citep*{Costabile.2004, Roberts.2008}.
Figure \ref{prototypeScreenshot} shows dashboard variant 6 in "comparison mode" comparing Hessen, Bayern and Berlin. At any time
the user can use the comparison control again to terminate the comparison process and all previously described changes are
reverted. The "dataset control" on the other hand allows the user to change the currently selected dataset via a button click.
After switching the dataset all views are rerendered with new labels and values.

\begin{figure}[ht]
    \centering
    \frame{\includegraphics[width=15.0cm]{images/prototype_screenshot.png}}
    \caption{Screenshot of the deployed web prototype. This shows dashbaord variant 6 with the \textit{highlighting\_2} interaction technique. The german states "Hessen", "Bayern" and "Berlin" are selected for comparison.} \label{prototypeScreenshot}
\end{figure}

\section{Multiple coordinated views}
As already stated earlier MCV systems are highly popular and can increase user performance. Baldonado et al. argue that they can provide utility
by minimizing cognitive overhead created from a single complex view. On the other hand, multiple views can also increase cognitive overhead
(e.g. context switching) and can raise system requirements. They propose guidelines to find out when to use MCV and how to use MCV
with a special focus on information visualization \citep*{WangBaldonado.2000}. One use of MCV in the prototype is the additional rendering of
two views when entering the "comparison mode". This mechanism represents the stated example from Baldonado et al. of using MCV to display
aggregates of the data. The rule of "Decomposition" motivated the use of additional views that aggregate differences when users are asked
about differences in the first place. As the name MCV already suggests \textit{coordination} is a really important factor when implementing
multiple views. Baldonado et al. devoted two guideline rules on this because multiple views if not coordinated correctly can create
confusion and increase computational overhead. Because of that, all our interaction techniques are synchronized across all views. Meaning if one
element is highlighted in one view, it is highlighted in all views. When entering the "comparison mode" the specific interaction technique
is also applied to all views. We also try to optimize screen space by replacing views when entering "comparison mode" or optionally rendering
the additional views that aggregate the differences. In that way, the "comparison mode" ensures that the additional data is only displayed when the
user is asking for it. We ensure that the use of more views does not noticeably increase
response time when interacting with the application. Finally, we use perceptual cues to indicate the replacement of views by animating
the timeline drawing on the initial render and by changing the labels of the views when changing the datasets. 
To further support the "Rule of consistency" we synchronize the "comparison control" and the "dataset control" across all six variants, to reduce
the computational overhead from context switching.

\section{Non-/Functional requirements}
This section describes all functional and non-functional requirements on the
web prototype. They are the result from the topics explained earlier in this chapter.

% \renewcommand{\arraystretch}{3} % Adjust the value as needed
\begin{longtable}{| p{0.05\linewidth} | p{0.65\linewidth} | p{0.20\linewidth}|}
    \hline
    \textbf{Nr.} & \textbf{Requirement} & \textbf{Type} \\
    \hline
    \endfirsthead
    \multicolumn{3}{l}{{\textit{Continued from previous page}}} \\
    \hline
    \textbf{Nr.} & \textbf{Requirement} & \textbf{Type} \\
    \hline
    \endhead
    \hline \multicolumn{3}{r}{{\textit{Continued on next page}}} \\
    \endfoot
    \hline
    \caption{This table describes all functional and non-functional requirements of the web prototype \label{requirement_table}}\\
    \endlastfoot
    01 & \parbox{\linewidth}{\vspace{4pt}The app should visualize one spatial-temporal dataset with polygons on a map. Each polygon should represent a spatial entity} & Functional\\
    \hline
    02 & \parbox{\linewidth}{\vspace{4pt}The app should allow the user to read the exact attribute value for every spatial entity over the whole time period in a separate view} & Functional\\
    \hline
    03 & \parbox{\linewidth}{\vspace{4pt}The app should visualize the temporal evolution for the attribute values of all spatial entities at a glance in a separate view} & Functional\\
    \hline
    04 & \parbox{\linewidth}{\vspace{4pt}The app eases the process of comparing two (or three) spatial entities through a comparison mode that can be switched on/off} & Functional\\
    \hline
    05 & \parbox{\linewidth}{\vspace{4pt}The app should provide six different dashboard versions that differ in their interaction technique and number of rendered views for comparing spatial entities} & Functional\\
    \hline
    06 & \parbox{\linewidth}{\vspace{4pt}The app consists of version 1 where only the selected entities are displayed. It represents a 'filtering' interaction technique} & Functional\\
    \hline
    07 & \parbox{\linewidth}{\vspace{4pt}The app consists of version 2 where the selected entities are visually emphasized using one color in comparison mode. It represents a 'highlighting' interaction technique} & Functional\\
    \hline
    08 & \parbox{\linewidth}{\vspace{4pt}The app consists of version 3 where the selected entities are visually emphasized using multiple colors in comparison mode. One color for each entity. It represents a 'highlighting' interaction technique} & Functional\\
    \hline
    09 & \parbox{\linewidth}{\vspace{4pt}The app consists of version 4 which is a copy of version 1 described in requirement 06. In addition, one more view that meets requirement 02 and one more view that meets requirement 03 are displayed in comparison mode. These additional views encode the subtracted values of the selected entities forming views that encode 'differences'. For every combination of the selected entities one additional data series is displayed.} & Functional\\
    \hline
    10 & \parbox{\linewidth}{\vspace{4pt}The app consists of version 5 which is a copy of version 2 described in requirement 07. In addition, one more view that meets requirement 02 and one more view that meets requirement 03 are displayed in comparison mode. These additional views encode the subtracted values of the selected entities forming views that encode 'differences'. For every combination of the selected entities one additional data series is displayed.} & Functional\\
    \hline
    11 & \parbox{\linewidth}{\vspace{4pt}The app consists of version 6 which is a copy of version 3 described in requirement 06. In addition, one more view that meets requirement 02 and one more view that meets requirement 03 are displayed in comparison mode. These additional views encode the subtracted values of the selected entities forming views that encode 'differences'. For every combination of the selected entities one additional data series is displayed. In line with version 6, which represents a highlighting interaction technique with individual colors, the rendered differences are also visually highlighted in different colors.} & Functional\\
    \hline
    12 & \parbox{\linewidth}{\vspace{4pt}In all six versions in comparison mode all selected spatial entities are highlighted on the map.} & Functional\\
    \hline
    13 & \parbox{\linewidth}{\vspace{4pt}The app enables the user to switch between four selected datasets which all have the same spatial-temporal dimensions} & Functional\\
    \hline
    14 & \parbox{\linewidth}{\vspace{4pt}The app should be available over a website} & Non-Functional\\
    \hline
    15 & \parbox{\linewidth}{\vspace{4pt}The app should be user-friendly and have fast loading times} & Non-Functional\\
    \hline
    16 & \parbox{\linewidth}{\vspace{4pt}The app should use multi-coordinated views appropriately by paying attention to common guidelines to reduce cognitive overhead.} & Non-Functional\\
\end{longtable}

\section{Presentation \&\ Deployment}
The code of the web prototype can be found on GitHub under the following link: \url{https://github.com/SimonMeissner/geo-dashboard}. A deployed
version of the prototype can be accessed by visiting: \url{https://geo-dashboard.vercel.app/}. In case the deployed version is no longer accessible
in the future a video recording is also deposited on the GitHub repository.
