\chapter{Digital Web-prototype}
The prototype was build using the open-source web development framework next.js which provides react-based javascript or
typescript applications with server-side rendering and static website generation. It was mostly selected because of its
developer-friendly ecosystem especially regarding deployement. Another reason for this stack was the existing experience
with react-based typescript and the light-weight and modern setup. The whole application makes use of a few react libraries.
The most important beeing the popular "react-leaflet" library which drastically simplifies using leaflet maps in react
applications. Other important libraries are: "zustand" to help with coordination, "recharts" to help building time lines and
"tailwindcss" to ease building the css of the application. Because the experiment was designed to be held online the
application needed to be deployed. For that we used the free deployment plan of vercel.com. Because the application only
used non-personal static data there was no need for implementing a backend or an authentication system. Although the prototype
took responsive behaviour into account the screen dimension of the user's device should not fall below 1280x720px. 
\section{Data}
spatial-temporal data
dataset could be changed using a Dataset Control
\section{Interaction techniques}
\section{Layout \&\ Function}
In addition to the static data, the entire application consisted of a simple frontend that served six different non-scrollable
geodashboard variants. On entering the page the first variant was rendered. The other variants were reachable over a
tab-based navigationbar at the top of the screen. A seventh tab was implemented to learn about the application and its use.
Each variant was covered with an openstreet map that was centered on Germany covering the rest of the page. All states of
Germany were rendered using the borders as polygon representations, building a cloropleth-map. Every polygon used the same
blue fill color. All other elements in the dashboards were placed on top of this map. On every variant, two additional views
were also rendered on top of each other, taking up the left third of the screen. The graph view in the top-left corner of the
screen visualized the spatial temporal data by rendering 16 lines, each representing one of the 16 states of germany. The
bottom-left view visualized a two-dimensional table connecting each state with each year containing a numerical value. The
dataset control and another button, which will be termed "comparison control" were positioned in the middle third of the
screen close to the navigationbar at the top. the comparison control offered the ability to select two or three of the 16
states for comparison. After selecting the states and starting the comparison process on three of the six variants an additional
graph view and an additional table view were rendered. In both views calculated differences between two states of the values
of the selected states were displayed. In the graph view again as lines representing the temporal evolution of the differences
and in the table view as simple numerical values for every year. Those are the already mentioned views that use
\textit{explicit encoding} as defined by Gleicher \citep*{Gleicher.2018}. After starting the comparison process both views
on the left are replaced which other views utilizing the different interaction techniques. \textit{Replace} is one of the
three common operational models visual interfaces use on parameter change \citep*{Costabile.2004, Roberts.2008}. At any time the user can
use the comparison control again to terminate the comparison process and all previously described changes are reverted.
\section{Coordination}
The whole prototype was designed with a focus on coordination and synchronization. 

\section{Non-/Functionl requirements}


% \renewcommand{\arraystretch}{3} % Adjust the value as needed
\begin{longtable}{| p{0.05\linewidth} | p{0.75\linewidth} | p{0.2\linewidth}|}
    \hline
    \textbf{Nr.} & \textbf{Requirement} & \textbf{Type} \\
    \hline
    \endfirsthead
    \multicolumn{3}{l}{{\textit{Continued from previous page}}} \\
    \hline
    \textbf{Nr.} & \textbf{Requirement} & \textbf{Type} \\
    \hline
    \endhead
    \hline \multicolumn{3}{r}{{\textit{Continued on next page}}} \\
    \endfoot
    \hline
    \caption{This table describes all functional- and nonfunctional requirements of the web-prototype\label{requirement_table}}\\
    \endlastfoot
    01 & \parbox{\linewidth}{\vspace{4pt}The app should visualize one spatial-temporal dataset with polygons on a map. Each polygon should represent a spatial entity} & Functional\\
    \hline
    02 & \parbox{\linewidth}{\vspace{4pt}The app should allow the user to read the exact attribute value for every spatial entity over the whole time period in a separate view} & Functional\\
    \hline
    03 & \parbox{\linewidth}{\vspace{4pt}The app should visualize the temporal evolution for the attribute values of all spatial entities at a glance in a separate view} & Functional\\
    \hline
    04 & \parbox{\linewidth}{\vspace{4pt}The app eases the process of comparing two (or three) spatial entities through a comparison mode that can be switched on/off} & Functional\\
    \hline
    05 & \parbox{\linewidth}{\vspace{4pt}The app should provide six different dashboard versions that differ in their interaction technique and number of rendered views for comparing spatial entities} & Functional\\
    \hline
    06 & \parbox{\linewidth}{\vspace{4pt}The app consists of version 1 where in comparison mode only the selected entities are filtered and shown in all views. It represents a 'filtering' interaction technique} & Functional\\
    \hline
    07 & \parbox{\linewidth}{\vspace{4pt}The app consists of version 2 where in comparison mode the selected entities are visually emphasized using one color. It represents a 'highlighting' interaction technique} & Functional\\
    \hline
    08 & \parbox{\linewidth}{\vspace{4pt}The app consists of version 3 where in comparison mode the selected entities are visually emphasized using multiple colors. One for each entity. It represents a 'highlighting' interaction technique} & Functional\\
    \hline
    09 & \parbox{\linewidth}{\vspace{4pt}The app consists of version 4 which is a copy of version 1 described in requirement 06. In addition, one more view that meets requirement 02 and one more view that meets requirement 03 are displayed in comparison mode. These additional views encode the subtracted values of the selected entities forming views that encode 'differences'. For every combination of the selected entities one additional data series is displayed.} & Functional\\
    \hline
    10 & \parbox{\linewidth}{\vspace{4pt}The app consists of version 5 which is a copy of version 2 described in requirement 07. In addition, one more view that meets requirement 02 and one more view that meets requirement 03 are displayed in comparison mode. These additional views encode the subtracted values of the selected entities forming views that encode 'differences'. For every combination of the selected entities one additional data series is displayed.} & Functional\\
    \hline
    11 & \parbox{\linewidth}{\vspace{4pt}The app consists of version 6 which is a copy of version 3 described in requirement 06. In addition, one more view that meets requirement 02 and one more view that meets requirement 03 are displayed in comparison mode. These additional views encode the subtracted values of the selected entities forming views that encode 'differences'. For every combination of the selected entities one additional data series is displayed. In line with version 6, which represents a highlighting interaction technique with individual colors, the rendered differences are also visually highlighted in different colors.} & Functional\\
    \hline
    12 & \parbox{\linewidth}{\vspace{4pt}In all six versions in comparison mode all selected spatial entities are highlighted on the map.} & Functional\\
    \hline
    13 & \parbox{\linewidth}{\vspace{4pt}The app enables the user to switch between four selected datasets which all have the same spatial-temporal dimensions} & Functional\\
    \hline
    14 & \parbox{\linewidth}{\vspace{4pt}The app should be available over a website} & Non-Functional\\
    \hline
    15 & \parbox{\linewidth}{\vspace{4pt}The app should be user-friendly and have fast loading times} & Non-Functional\\
    \hline
    16 & \parbox{\linewidth}{\vspace{4pt}The app should use multi-coordinated views appropriately by paying attention to common guidelines to reduce cognitive overhead.} & Non-Functional\\
\end{longtable}